%% Support sites:
%% http://www.michaelshell.org/tex/ieeetran/
%% http://www.ctan.org/tex-archive/macros/latex/contrib/IEEEtran/
%% and
%% http://www.ieee.org/

%%*************************************************************************
%% Legal Notice:
%% This code is offered as-is without any warranty either expressed or
%% implied; without even the implied warranty of MERCHANTABILITY or
%% FITNESS FOR A PARTICULAR PURPOSE! 
%% User assumes all risk.
%% In no event shall IEEE or any contributor to this code be liable for
%% any damages or losses, including, but not limited to, incidental,
%% consequential, or any other damages, resulting from the use or misuse
%% of any information contained here.
%%
%% All comments are the opinions of their respective authors and are not
%% necessarily endorsed by the IEEE.
%%
%% This work is distributed under the LaTeX Project Public License (LPPL)
%% ( http://www.latex-project.org/ ) version 1.3, and may be freely used,
%% distributed and modified. A copy of the LPPL, version 1.3, is included
%% in the base LaTeX documentation of all distributions of LaTeX released
%% 2003/12/01 or later.
%% Retain all contribution notices and credits.
%% ** Modified files should be clearly indicated as such, including  **
%% ** renaming them and changing author support contact information. **
%%
%% File list of work: IEEEtran.cls, IEEEtran_HOWTO.pdf, bare_adv.tex,
%%                    bare_conf.tex, bare_jrnl.tex, bare_jrnl_compsoc.tex
%%*************************************************************************

% *** Authors should verify (and, if needed, correct) their LaTeX system  ***
% *** with the testflow diagnostic prior to trusting their LaTeX platform ***
% *** with production work. IEEE's font choices can trigger bugs that do  ***
% *** not appear when using other class files.                            ***
% The testflow support page is at:
% http://www.michaelshell.org/tex/testflow/



% Note that the a4paper option is mainly intended so that authors in
% countries using A4 can easily print to A4 and see how their papers will
% look in print - the typesetting of the document will not typically be
% affected with changes in paper size (but the bottom and side margins will).
% Use the testflow package mentioned above to verify correct handling of
% both paper sizes by the user's LaTeX system.
%
% Also note that the "draftcls" or "draftclsnofoot", not "draft", option
% should be used if it is desired that the figures are to be displayed in
% draft mode.
%
\documentclass[conference]{IEEEtran}
% *** CITATION PACKAGES ***
%
%\usepackage{cite}
% cite.sty was written by Donald Arseneau
% V1.6 and later of IEEEtran pre-defines the format of the cite.sty package
% \cite{} output to follow that of IEEE. Loading the cite package will
% result in citation numbers being automatically sorted and properly
% "compressed/ranged". e.g., [1], [9], [2], [7], [5], [6] without using
% cite.sty will become [1], [2], [5]--[7], [9] using cite.sty. cite.sty's
% \cite will automatically add leading space, if needed. Use cite.sty's
% noadjust option (cite.sty V3.8 and later) if you want to turn this off.
% cite.sty is already installed on most LaTeX systems. Be sure and use
% version 4.0 (2003-05-27) and later if using hyperref.sty. cite.sty does
% not currently provide for hyperlinked citations.
% The latest version can be obtained at:
% http://www.ctan.org/tex-archive/macros/latex/contrib/cite/
% The documentation is contained in the cite.sty file itself.






% *** GRAPHICS RELATED PACKAGES ***
%
\ifCLASSINFOpdf
  % \usepackage[pdftex]{graphicx}
  % declare the path(s) where your graphic files are
  % \graphicspath{{../pdf/}{../jpeg/}}
  % and their extensions so you won't have to specify these with
  % every instance of \includegraphics
  % \DeclareGraphicsExtensions{.pdf,.jpeg,.png}
\else
  % or other class option (dvipsone, dvipdf, if not using dvips). graphicx
  % will default to the driver specified in the system graphics.cfg if no
  % driver is specified.
  % \usepackage[dvips]{graphicx}
  % declare the path(s) where your graphic files are
  % \graphicspath{{../eps/}}
  % and their extensions so you won't have to specify these with
  % every instance of \includegraphics
  % \DeclareGraphicsExtensions{.eps}
\fi
% graphicx was written by David Carlisle and Sebastian Rahtz. It is
% required if you want graphics, photos, etc. graphicx.sty is already
% installed on most LaTeX systems. The latest version and documentation can
% be obtained at: 
% http://www.ctan.org/tex-archive/macros/latex/required/graphics/
% Another good source of documentation is "Using Imported Graphics in
% LaTeX2e" by Keith Reckdahl which can be found as epslatex.ps or
% epslatex.pdf at: http://www.ctan.org/tex-archive/info/
%
% latex, and pdflatex in dvi mode, support graphics in encapsulated
% postscript (.eps) format. pdflatex in pdf mode supports graphics
% in .pdf, .jpeg, .png and .mps (metapost) formats. Users should ensure
% that all non-photo figures use a vector format (.eps, .pdf, .mps) and
% not a bitmapped formats (.jpeg, .png). IEEE frowns on bitmapped formats
% which can result in "jaggedy"/blurry rendering of lines and letters as
% well as large increases in file sizes.
%
% You can find documentation about the pdfTeX application at:
% http://www.tug.org/applications/pdftex





% *** MATH PACKAGES ***
%
\usepackage[cmex10]{amsmath}%
% Also, note that the amsmath package sets \interdisplaylinepenalty to 10000
% thus preventing page breaks from occurring within multiline equations. Use:
%\interdisplaylinepenalty=2500
% after loading amsmath to restore such page breaks as IEEEtran.cls normally
% does. amsmath.sty is already installed on most LaTeX systems. The latest
% version and documentation can be obtained at:
% http://www.ctan.org/tex-archive/macros/latex/required/amslatex/math/





% *** SPECIALIZED LIST PACKAGES ***
%
%\usepackage{algorithmic}
% algorithmic.sty was written by Peter Williams and Rogerio Brito.
% This package provides an algorithmic environment fo describing algorithms.
% You can use the algorithmic environment in-text or within a figure
% environment to provide for a floating algorithm. Do NOT use the algorithm
% floating environment provided by algorithm.sty (by the same authors) or
% algorithm2e.sty (by Christophe Fiorio) as IEEE does not use dedicated
% algorithm float types and packages that provide these will not provide
% correct IEEE style captions. The latest version and documentation of
% algorithmic.sty can be obtained at:
% http://www.ctan.org/tex-archive/macros/latex/contrib/algorithms/
% There is also a support site at:
% http://algorithms.berlios.de/index.html
% Also of interest may be the (relatively newer and more customizable)
% algorithmicx.sty package by Szasz Janos:
% http://www.ctan.org/tex-archive/macros/latex/contrib/algorithmicx/




% *** ALIGNMENT PACKAGES ***
%
%\usepackage{array}
% Frank Mittelbach's and David Carlisle's array.sty patches and improves
% the standard LaTeX2e array and tabular environments to provide better
% appearance and additional user controls. As the default LaTeX2e table
% generation code is lacking to the point of almost being broken with
% respect to the quality of the end results, all users are strongly
% advised to use an enhanced (at the very least that provided by array.sty)
% set of table tools. array.sty is already installed on most systems. The
% latest version and documentation can be obtained at:
% http://www.ctan.org/tex-archive/macros/latex/required/tools/


% IEEEtran contains the IEEEeqnarray family of commands that can be used to
% generate multiline equations as well as matrices, tables, etc., of high
% quality.


%\usepackage{eqparbox}
% Also of notable interest is Scott Pakin's eqparbox package for creating
% (automatically sized) equal width boxes - aka "natural width parboxes".
% Available at:
% http://www.ctan.org/tex-archive/macros/latex/contrib/eqparbox/


% *** PDF, URL AND HYPERLINK PACKAGES ***
%
%\usepackage{url}
% url.sty was written by Donald Arseneau. It provides better support for
% handling and breaking URLs. url.sty is already installed on most LaTeX
% systems. The latest version can be obtained at:
% http://www.ctan.org/tex-archive/macros/latex/contrib/misc/
% Read the url.sty source comments for usage information. Basically,
% \url{my_url_here}.





% correct bad hyphenation here
\hyphenation{op-tical net-works semi-conduc-tor}


\begin{document}
%
% paper title
% can use linebreaks \\ within to get better formatting as desired
\title{River Runoff Forecasting}


% author names and affiliations
% use a multiple column layout for up to three different
% affiliations
\author{\IEEEauthorblockN{Brian Bell, Brian Wallace, Du Zhang}
\IEEEauthorblockA{
California State University, Sacramento\\
Department of Computer Science
Sacramento, California\\
Email: bryan.w.bell@gmail.com, bwtech@gmail.com, zhangd@ecs.csus.edu}
}

% make the title area
\maketitle


\begin{abstract}
%\boldmath
How "wet" or "dry" a year is predicted to be has many impacts. Public utilities
need to determine what percentage of their electric energy generation will be
hydro power. Good water years enable the utilities to use more hydro power
and, consequently, save oil. Conversely, in a dry year, the utilities must depend
more on steam generation and therefore use more oil, coal, and atomic fuel.
Agricultural interest use the information to determine crop planting patterns,
ground water pumping needs, and irrigation schedules. Operators of flood
control projects determine how much water can safely be stored in a reservoir
while reserving space for predicted inflows. Municipalities use the information to
evaluate their water supply and determine whether (in a dry year) water rationing
my be needed.

\end{abstract}
% IEEEtran.cls defaults to using nonbold math in the Abstract.
% This preserves the distinction between vectors and scalars. However,
% if the conference you are submitting to favors bold math in the abstract,
% then you can use LaTeX's standard command \boldmath at the very start
% of the abstract to achieve this. Many IEEE journals/conferences frown on
% math in the abstract anyway.

% no keywords




% For peer review papers, you can put extra information on the cover
% page as needed:
% \ifCLASSOPTIONpeerreview
% \begin{center} \bfseries EDICS Category: 3-BBND \end{center}
% \fi
%
% For peerreview papers, this IEEEtran command inserts a page break and
% creates the second title. It will be ignored for other modes.
\IEEEpeerreviewmaketitle



\section{Introduction}
% no \IEEEPARstart

Water forecasts lead to better planning and management of the State's water
resources -- which benefit all Californians. The Cooperative Snow Surveys
Program is an important part of this effort. Thus, Californians are dependent
upon snow . . . and the snow surveyor.


Today in California more than 50 state, national, and private agencies pool
their efforts in collecting snow data. Over three hundred snow courses (http:/
/cdec.water.ca.gov/cgi-progs/snowsurvey\_p/SNOWTAB6) are sampled each
winter.

One of the forecasting products produced by Snow Surveys is the Bulletin
120 (http://cdec.water.ca.gov/snow/bulletin120/index2.html . Bulletin 120 is a
publication issued four times a year, in the second week of February, March,
April, and May by the California Department of Water Resources. It contains
forecasts of the volume of seasonal runoff from the state's major watersheds,
and summaries of precipitation, snowpack, reservoir storage, and runoff in
various regions of the State.

Our project focused on a sub-section of the Bulletin 120, the American
River. We forecasted the April – July “full natural flow” runoff of the American
River measured at Folsom (http://cdec.water.ca.gov/cgi-progs/staMeta?
station\_id=AMF).

\section{Approach and Learning Algorithms}
We used monthly precipitation and snow data gathered from 10 precipitation
monitoring stations and 28 snow monitoring stations located in the American
River basin. We also made use of the historical full natural flow data for the
American River at Folsom.

We used two learning algorithm(s). The first method consisted of feeding the
historical data into a neural-net and using the resulting neural-net to create
forecasts.
The second method consisted of creating a linear regression equation. The
regression equation is of the form:

river\_flow = a * station1[oct] + b * station1[nov] + ... + z * station2[oct] + b *
station2[nov] + ... +...

Where a,b,... are the coefficients that weight the station inputs by their relevance
to the final river\_flow.


\subsection{Related Work}
In searching for related work we found many articles about using neural networks
for water supply and stream flow prediction. Most of the articles focused on short-
term changes in stream flow, for example predicting flow after a heavy storm.

Interestingly, we didn’t find any articles about stream flow forecasts by neural
networks concerning basins in California. We’ve listed some of the related
papers:

[1] Kuo, Chun-Chao, Thian Yew Gan, and Pao-Shan Yu. "Seasonal Streamflow
Prediction by a Combined Climate-hydrologic System for River Basins of
Taiwan." Journal of Hydrology, 387.3/4 (2010): 292-303.
[2] Kentel, Elcin. "Estimation of River Flow by Artificial Neural Networks and
Identification of Input Vectors Susceptible to Producing Unreliable Flow
Estimates." Journal of Hydrology, 375.3/4 (2009): 481-488.
[3] Besaw, Lance, Donna Rizzo, Paul Bierman, and William Hackett. "Advances in

Ungauged Streamflow Prediction Using Artificial Neural Networks." Journal of
Hydrology, 386.1-4 (2010): 27-24.
[4] Liu, Fang, Jian-Zhong Zhou, Fang-Peng Qiu, and Jun-Jie Yang. "Biased Wavelet
Neural Network and Its Application to Streamflow Forecast." Lecture Notes in
Computer Science, 3971.2006 (2006): .
[5] Makkeasorn, A, N.B Chang, and X Zhou. "Short-term Streamflow Forecasting with
Global Climate Change Implications - a Comparative Study Between Genetic
Programming and Neural Network Models." Journal of Hydrology, 352.3-4 (2008):
336-354.

\section{Design of Learning Application}
For the design of our learning tool we used WEKA. WEKA enabled us to
completely forgo any algorithm coding and focus strictly on the set of algorithms
and algorithm parameters we used for our project. Most of the work consisted
of formatting the raw data and filtering into a set of attributes and instances that
were amenable to use by the neural nets.

\section{Experimental Results}
For the neural net we started out with no filtering of the data and over 500 input
parameters consisting of precipitation and snow water content entries. The
results were extremely disappointing.

We significantly filtered out some of the parameters that had sparse data and
trimmed the number of input parameters for the neural network down to 222. Our
results were better with the smaller number of parameters, but we still couldn’t
beet the human ensemble. During drought years the neural network produced
particularly bad results.

We made another attempt at the neural network, but this time only using five
parameters for input. The input parameters used were the indices produced from
the precipitation data and the snow water content data that the forecasters use
for input to their regression equations. This neural network showed much more
promise. Our results were much closer to matching the human ensemble. Our
relative absolute error with this neural network was actually smaller, however we
still couldn’t match the humans in terms of the other error measurements: mean
absolute error, root mean squared error, and root relative squared error.
The below training session has the results of training neural net1 with 222
attributes. Please note the error rate is low because we did not run it on a
forecast but instead ran it on the full data set.

=== Run information ===

Scheme:
weka.classifiers.functions.MultilayerPerceptron -L 0.03 -M 0.1 -N
2000 -V 0 -S 0 -E 20 -H "180, 150, 100" -G -R
Relation: AmericanRiv\_Train-weka.filters.unsupervised.attribute.Remove-R1
Instances: 110
Attributes: 222
[list of attributes omitted]
Test mode: user supplied test set: size unknown (reading incrementally)

=== Predictions on test set ===
inst\#, actual, predicted, error
1 1108933 1099143.411 -9789.589
2 552626 566670.998 14044.998

3 973817 994475.684 20658.684
4 1354434 1364647.994 10213.994
5 632159 649646.758 17487.758
6 2003878 1983893.455 -19984.545
7 2622387 2617264.94 -5122.06
8 522651 511853.949 -10797.051
9 674287 664542.387 -9744.613
10 1068327 1071307.545 2980.545
11 1486780 1475169.222 -11610.778

=== Evaluation on test set ===
=== Summary ===

Correlation coefficient
Mean absolute error
Root mean squared error
Relative absolute error
Root relative squared error
Total Number of Instances

0.9998
12039.5105
13180.1219
2.2054 \%
2.0253 \%
11

\section{Performance Evaluation and Comparison}
To evaluate our performance we compare the two neural nets, two linear
regressions to each other and importantly also to the human ensemble forecast.
In our case the comparison we are most interested in is the comparison of each
machine learning algorithm with the human ensemble.

\subsection{Comparison of Neural Net1 (222 inputs, no filtering of data) with Human
Ensemble:}
For this case the human ensemble forecast performed significantly better than
the neural net. The error rates were 69% and 83% respectively for the human
ensemble and the neural net.

\section{Description of development tools/methodologies used}
For our learning algorithms we used WEKA. The data was obtained from the
Oracle database that the Department of Water Resources maintains which has
the precipitation data for the state of California. We extracted the relevant data
using SQL from the database and then used the CSV to ARFF converter to
convert it into an ARFF file.

\section{Critique of Learning Algorithms Used}
The neural net did not perform well on the outlying years, either the extremely
wet years or the extremely dry years. In both cases the humans also did well but
they did better than the neural nets.
The linear regression performed extremely well on normal years but also
performed more poorly on outlying years.

% An example of a floating figure using the graphicx package.
% Note that \label must occur AFTER (or within) \caption.
% For figures, \caption should occur after the \includegraphics.
% Note that IEEEtran v1.7 and later has special internal code that
% is designed to preserve the operation of \label within \caption
% even when the captionsoff option is in effect. However, because
% of issues like this, it may be the safest practice to put all your
% \label just after \caption rather than within \caption{}.
%
% Reminder: the "draftcls" or "draftclsnofoot", not "draft", class
% option should be used if it is desired that the figures are to be
% displayed while in draft mode.
%
%\begin{figure}[!t]
%\centering
%\includegraphics[width=2.5in]{myfigure}
% where an .eps filename suffix will be assumed under latex, 
% and a .pdf suffix will be assumed for pdflatex; or what has been declared
% via \DeclareGraphicsExtensions.
%\caption{Simulation Results}
%\label{fig_sim}
%\end{figure}

% Note that IEEE typically puts floats only at the top, even when this
% results in a large percentage of a column being occupied by floats.


% An example of a double column floating figure using two subfigures.
% (The subfig.sty package must be loaded for this to work.)
% The subfigure \label commands are set within each subfloat command, the
% \label for the overall figure must come after \caption.
% \hfil must be used as a separator to get equal spacing.
% The subfigure.sty package works much the same way, except \subfigure is
% used instead of \subfloat.
%
%\begin{figure*}[!t]
%\centerline{\subfloat[Case I]\includegraphics[width=2.5in]{subfigcase1}%
%\label{fig_first_case}}
%\hfil
%\subfloat[Case II]{\includegraphics[width=2.5in]{subfigcase2}%
%\label{fig_second_case}}}
%\caption{Simulation results}
%\label{fig_sim}
%\end{figure*}
%
% Note that often IEEE papers with subfigures do not employ subfigure
% captions (using the optional argument to \subfloat), but instead will
% reference/describe all of them (a), (b), etc., within the main caption.


% An example of a floating table. Note that, for IEEE style tables, the 
% \caption command should come BEFORE the table. Table text will default to
% \footnotesize as IEEE normally uses this smaller font for tables.
% The \label must come after \caption as always.
%
%\begin{table}[!t]
%% increase table row spacing, adjust to taste
%\renewcommand{\arraystretch}{1.3}
% if using array.sty, it might be a good idea to tweak the value of
% \extrarowheight as needed to properly center the text within the cells
%\caption{An Example of a Table}
%\label{table_example}
%\centering
%% Some packages, such as MDW tools, offer better commands for making tables
%% than the plain LaTeX2e tabular which is used here.
%\begin{tabular}{|c||c|}
%\hline
%One & Two\\
%\hline
%Three & Four\\
%\hline
%\end{tabular}
%\end{table}


% Note that IEEE does not put floats in the very first column - or typically
% anywhere on the first page for that matter. Also, in-text middle ("here")
% positioning is not used. Most IEEE journals/conferences use top floats
% exclusively. Note that, LaTeX2e, unlike IEEE journals/conferences, places
% footnotes above bottom floats. This can be corrected via the \fnbelowfloat
% command of the stfloats package.



\section{Conclusion}
Our current results with our best neural nets yields error rates that are
comparable to the human ensemble forecast. The linear regression results
were very promising and more investigation should be completed before we can
conclude that it beat the human ensemble.

Our most promising line of future work is two use neural net ensemble instead of
using a single neural net for producing forecast. The ensemble would consist of
three neural nets that are trained and dry, normal, and wet years respectively.




% conference papers do not normally have an appendix


% use section* for acknowledgement
\section*{Acknowledgment}


The authors would like to thank...





% trigger a \newpage just before the given reference
% number - used to balance the columns on the last page
% adjust value as needed - may need to be readjusted if
% the document is modified later
%\IEEEtriggeratref{8}
% The "triggered" command can be changed if desired:
%\IEEEtriggercmd{\enlargethispage{-5in}}

% references section

% can use a bibliography generated by BibTeX as a .bbl file
% BibTeX documentation can be easily obtained at:
% http://www.ctan.org/tex-archive/biblio/bibtex/contrib/doc/
% The IEEEtran BibTeX style support page is at:
% http://www.michaelshell.org/tex/ieeetran/bibtex/
%\bibliographystyle{IEEEtran}
% argument is your BibTeX string definitions and bibliography database(s)
%\bibliography{IEEEabrv,../bib/paper}
%
% <OR> manually copy in the resultant .bbl file
% set second argument of \begin to the number of references
% (used to reserve space for the reference number labels box)
\begin{thebibliography}{1}

\bibitem{IEEEhowto:kopka}
H.~Kopka and P.~W. Daly, \emph{A Guide to \LaTeX}, 3rd~ed.\hskip 1em plus
  0.5em minus 0.4em\relax Harlow, England: Addison-Wesley, 1999.

\end{thebibliography}




% that's all folks
\end{document}


